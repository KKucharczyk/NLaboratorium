\documentclass[a4paper,12pt]{article}

\usepackage[utf8]{inputenc}
\usepackage[T1]{fontenc}
\usepackage[MeX]{polski}
\usepackage[polish]{babel}
\usepackage[latin2]{inputenc}
\usepackage[T1]{fontenc}
\usepackage{graphicx}
\usepackage{subfig}

\begin{document}
{\raggedleft{}Krzysztof Kucharczyk}\\200401\\Wydział Elektroniki\\Kierunek AiR
\\PAMSI lab. czw. 10:00-13:15\\\\\\
\begin{center} 
	\textbf{Sprawozdanie z laboratorium nr 10\\(Problem plecakowy)}
\end{center}

\section{Opis zadania}

Zadanie polegało na zaimplementowaniu algorytmu umożliwiającego rozwiązanie problemu plecakowego.
\section{Realizacja}

Postawiony problem postanowiłem rozwiązać za pomocą dostępnych z podstawowej biblioteki STL obiektów takich jak wektor oraz para. Umożliwiły mi one w prosty sposób poradzenie sobie z problemem. Wszystkie możliwe przedmioty oraz kombinacje optymalnych przedmiotów przechowywane są w wektorach, ze względu na ich uniwersalność i możliwość kustomizacji wedle potrzeb. Same przedmioty zaimplementowałem jako połączenie wagi i wartości.

\section{Testowanie}

Testowanie poprawności programu ograniczyłem do sprawdzenia poprawności dla kilku zestawu danych, np. dla danych zebranych w tabeli poniżej (pojemność plecaka = 10):

\begin{table}[ht] 
\caption{Dane testowe} 
\centering 
\begin{tabular}{|c|c|}

\hline
\hline
2 & 3 \\
\hline
4 & 1 \\
\hline
9 & 5 \\
\hline
2 & 4 \\
\hline
7 & 1 \\
\hline
3 & 4 \\
\hline
1 & 0 \\
\hline
8 & 3 \\
\hline
4 & 1 \\
\hline
4 & 6 \\
\hline \hline

\end{tabular}
\end{table}

Otrzymane wyniki wyglądają następująco:

\begin{table}[ht] 
\caption{Wyniki} 
\centering 
\begin{tabular}{|c|c|}

\hline
\hline
4 & 6 \\
\hline
3 & 4 \\
\hline
2 & 5 \\
\hline \hline

\end{tabular}
\end{table}
\newpage
\section{Wnioski}

Otrzymane wyniki wydają się być wiarygodne. Prosta implementacja pomogła uniknąć błędów. Zastosowanie tablicy działajacej na zasadzie postulatów programowania dynamicznego pozwoliło w prosty sposób rozwiązać problem plecakowy. 
\end{document}



