\documentclass[a4paper,12pt]{article}

\usepackage[utf8]{inputenc}
\usepackage[T1]{fontenc}
\usepackage[MeX]{polski}
\usepackage[polish]{babel}
\usepackage[latin2]{inputenc}
\usepackage{graphicx}
\usepackage{epstopdf}
\usepackage[T1]{fontenc}

\begin{document}
Krzysztof Kucharczyk\\200401\\Wydział Elektroniki\\Kierunek AiR
\\PAMSI lab. czw. 10:00-13:15\\\\\\
\begin{center} 
Sprawozdanie z laboratorium nr 3\\(Listy i stosy)
\end{center}

\section{Opis zadania}

Zadanie polegało na stworzeniu pięciu algorytmów, tj. 
\begin{enumerate} 
\item Stosu na bazie tablicy (powiększanie tablicy o jedno miejsce)
\item Stosu na bazie tablicy (podwajanie tablicy)
\item Stosu na bazie listy jednokierunkowej
\item Kolejki na bazie tablicy
\item Kolejki na bazie listy jednokierunkowej

\end{enumerate}

oraz przetestowaniu wydajności dwóch pierwszych algorytmów.

\subsection{Opis algorytmu 1}

Algorytm ten miał działać jak zwykły stos. Problem przepełnienia stosu 
rozwiązany jest poprzez tworzenie nowej, większej tablicy, do której 
przepisywana jest stara tablica, a następnie usuwana. Nowa tablica zawiera
jedno miejsce dodatkowo, w które wpisywany zostaje nowy element.

Usuwanie elementu z listy następuje poprzez stworzenie mniejszej tablicy, 
do której przepisywane są elementy starej poza ostatnim, następnie stara
tablica zostaje usunięta i nadpisana nową.

\subsection{Opis algorytmu 2}

Algorytm zbudowany jest bardzo podobnie do poprzedniego, jednak tym razem 
tablica zostaje powiększona dwukrotnie w przypadku przepełnienia, natomiast
zmniejszana o połowę w przypadku gdy ilość danych wynosi ćwierć wielkości
tablicy.\\

Oba algorytmy posiadają identyczny sposób mierzenia czasu, tj. pomiar odbywa
 się za pośrednictwem odpowiedniej biblioteki, która następnie tworzy
tymczasowy plik, w którym przechowywane są kolejne wyniki pomiarów. Program
./benchmark wyznacza średnią z owych pomiarów i umieszcza ją w pliku 
wynikowym.

\section{Pomiary wydajności algorytmów}

Na następnej stronie zamieszczone są wykresy wydajności algorytmów 1 i 2, zgodnie z przjętą
powyżej konwencją. Wykres algorytmu 1 (zwiększający tablicę o jeden) ewidentnie
szybciej rośnie w górę, przez co okazuje się być gorszym algorytmem. Z kolei
algorytm 2 (zwiększający dwukrotnie wielkość tablicy) jest niemalże niewidoczny,
gdyż znajduje się bardzo blisko osi rzędnych, czyli nie potrzebuje dużego
czasu, aby uporać się ze sporą ilością danych. Dla prostszej analizy wstawiłem
wykresy dla dwóch zakresów, aby wykres drugi był nieco bardziej widoczny.

\begin{figure}[hb]
% GNUPLOT: LaTeX picture
\setlength{\unitlength}{0.240900pt}
\ifx\plotpoint\undefined\newsavebox{\plotpoint}\fi
\sbox{\plotpoint}{\rule[-0.200pt]{0.400pt}{0.400pt}}%
\begin{picture}(1500,900)(0,0)
\sbox{\plotpoint}{\rule[-0.200pt]{0.400pt}{0.400pt}}%
\put(231.0,131.0){\rule[-0.200pt]{291.007pt}{0.400pt}}
\put(231.0,131.0){\rule[-0.200pt]{4.818pt}{0.400pt}}
\put(211,131){\makebox(0,0)[r]{ 0}}
\put(1419.0,131.0){\rule[-0.200pt]{4.818pt}{0.400pt}}
\put(231.0,260.0){\rule[-0.200pt]{291.007pt}{0.400pt}}
\put(231.0,260.0){\rule[-0.200pt]{4.818pt}{0.400pt}}
\put(211,260){\makebox(0,0)[r]{ 0.0005}}
\put(1419.0,260.0){\rule[-0.200pt]{4.818pt}{0.400pt}}
\put(231.0,389.0){\rule[-0.200pt]{291.007pt}{0.400pt}}
\put(231.0,389.0){\rule[-0.200pt]{4.818pt}{0.400pt}}
\put(211,389){\makebox(0,0)[r]{ 0.001}}
\put(1419.0,389.0){\rule[-0.200pt]{4.818pt}{0.400pt}}
\put(231.0,518.0){\rule[-0.200pt]{291.007pt}{0.400pt}}
\put(231.0,518.0){\rule[-0.200pt]{4.818pt}{0.400pt}}
\put(211,518){\makebox(0,0)[r]{ 0.0015}}
\put(1419.0,518.0){\rule[-0.200pt]{4.818pt}{0.400pt}}
\put(231.0,647.0){\rule[-0.200pt]{204.283pt}{0.400pt}}
\put(1419.0,647.0){\rule[-0.200pt]{4.818pt}{0.400pt}}
\put(231.0,647.0){\rule[-0.200pt]{4.818pt}{0.400pt}}
\put(211,647){\makebox(0,0)[r]{ 0.002}}
\put(1419.0,647.0){\rule[-0.200pt]{4.818pt}{0.400pt}}
\put(231.0,776.0){\rule[-0.200pt]{291.007pt}{0.400pt}}
\put(231.0,776.0){\rule[-0.200pt]{4.818pt}{0.400pt}}
\put(211,776){\makebox(0,0)[r]{ 0.0025}}
\put(1419.0,776.0){\rule[-0.200pt]{4.818pt}{0.400pt}}
\put(341.0,131.0){\rule[-0.200pt]{0.400pt}{155.380pt}}
\put(341.0,131.0){\rule[-0.200pt]{0.400pt}{4.818pt}}
\put(341,90){\makebox(0,0){ 100}}
\put(341.0,756.0){\rule[-0.200pt]{0.400pt}{4.818pt}}
\put(463.0,131.0){\rule[-0.200pt]{0.400pt}{155.380pt}}
\put(463.0,131.0){\rule[-0.200pt]{0.400pt}{4.818pt}}
\put(463,90){\makebox(0,0){ 200}}
\put(463.0,756.0){\rule[-0.200pt]{0.400pt}{4.818pt}}
\put(585.0,131.0){\rule[-0.200pt]{0.400pt}{155.380pt}}
\put(585.0,131.0){\rule[-0.200pt]{0.400pt}{4.818pt}}
\put(585,90){\makebox(0,0){ 300}}
\put(585.0,756.0){\rule[-0.200pt]{0.400pt}{4.818pt}}
\put(707.0,131.0){\rule[-0.200pt]{0.400pt}{155.380pt}}
\put(707.0,131.0){\rule[-0.200pt]{0.400pt}{4.818pt}}
\put(707,90){\makebox(0,0){ 400}}
\put(707.0,756.0){\rule[-0.200pt]{0.400pt}{4.818pt}}
\put(829.0,131.0){\rule[-0.200pt]{0.400pt}{155.380pt}}
\put(829.0,131.0){\rule[-0.200pt]{0.400pt}{4.818pt}}
\put(829,90){\makebox(0,0){ 500}}
\put(829.0,756.0){\rule[-0.200pt]{0.400pt}{4.818pt}}
\put(951.0,131.0){\rule[-0.200pt]{0.400pt}{155.380pt}}
\put(951.0,131.0){\rule[-0.200pt]{0.400pt}{4.818pt}}
\put(951,90){\makebox(0,0){ 600}}
\put(951.0,756.0){\rule[-0.200pt]{0.400pt}{4.818pt}}
\put(1073.0,131.0){\rule[-0.200pt]{0.400pt}{155.380pt}}
\put(1073.0,131.0){\rule[-0.200pt]{0.400pt}{4.818pt}}
\put(1073,90){\makebox(0,0){ 700}}
\put(1073.0,756.0){\rule[-0.200pt]{0.400pt}{4.818pt}}
\put(1195.0,131.0){\rule[-0.200pt]{0.400pt}{120.932pt}}
\put(1195.0,756.0){\rule[-0.200pt]{0.400pt}{4.818pt}}
\put(1195.0,131.0){\rule[-0.200pt]{0.400pt}{4.818pt}}
\put(1195,90){\makebox(0,0){ 800}}
\put(1195.0,756.0){\rule[-0.200pt]{0.400pt}{4.818pt}}
\put(1317.0,131.0){\rule[-0.200pt]{0.400pt}{120.932pt}}
\put(1317.0,756.0){\rule[-0.200pt]{0.400pt}{4.818pt}}
\put(1317.0,131.0){\rule[-0.200pt]{0.400pt}{4.818pt}}
\put(1317,90){\makebox(0,0){ 900}}
\put(1317.0,756.0){\rule[-0.200pt]{0.400pt}{4.818pt}}
\put(1439.0,131.0){\rule[-0.200pt]{0.400pt}{155.380pt}}
\put(1439.0,131.0){\rule[-0.200pt]{0.400pt}{4.818pt}}
\put(1439,90){\makebox(0,0){ 1000}}
\put(1439.0,756.0){\rule[-0.200pt]{0.400pt}{4.818pt}}
\put(231.0,131.0){\rule[-0.200pt]{0.400pt}{155.380pt}}
\put(231.0,131.0){\rule[-0.200pt]{291.007pt}{0.400pt}}
\put(1439.0,131.0){\rule[-0.200pt]{0.400pt}{155.380pt}}
\put(231.0,776.0){\rule[-0.200pt]{291.007pt}{0.400pt}}
\put(30,453){\makebox(0,0){Czas w [s]}}
\put(835,29){\makebox(0,0){Wielkosc problemy (ilosc danych)}}
\put(835,838){\makebox(0,0){Wykres wydajnosci algorytmow}}
\put(1249,735){\makebox(0,0){Legenda}}
\put(1279,695){\makebox(0,0)[r]{Algorytm 1}}
\put(1299.0,695.0){\rule[-0.200pt]{24.090pt}{0.400pt}}
\put(231,144){\usebox{\plotpoint}}
\put(231,143.67){\rule{2.891pt}{0.400pt}}
\multiput(231.00,143.17)(6.000,1.000){2}{\rule{1.445pt}{0.400pt}}
\put(255,144.67){\rule{3.132pt}{0.400pt}}
\multiput(255.00,144.17)(6.500,1.000){2}{\rule{1.566pt}{0.400pt}}
\put(268,145.67){\rule{2.891pt}{0.400pt}}
\multiput(268.00,145.17)(6.000,1.000){2}{\rule{1.445pt}{0.400pt}}
\put(243.0,145.0){\rule[-0.200pt]{2.891pt}{0.400pt}}
\put(292,146.67){\rule{2.891pt}{0.400pt}}
\multiput(292.00,146.17)(6.000,1.000){2}{\rule{1.445pt}{0.400pt}}
\put(304,147.67){\rule{2.891pt}{0.400pt}}
\multiput(304.00,147.17)(6.000,1.000){2}{\rule{1.445pt}{0.400pt}}
\put(316,148.67){\rule{3.132pt}{0.400pt}}
\multiput(316.00,148.17)(6.500,1.000){2}{\rule{1.566pt}{0.400pt}}
\put(329,150.17){\rule{2.500pt}{0.400pt}}
\multiput(329.00,149.17)(6.811,2.000){2}{\rule{1.250pt}{0.400pt}}
\put(341,151.67){\rule{2.891pt}{0.400pt}}
\multiput(341.00,151.17)(6.000,1.000){2}{\rule{1.445pt}{0.400pt}}
\put(353,153.17){\rule{2.500pt}{0.400pt}}
\multiput(353.00,152.17)(6.811,2.000){2}{\rule{1.250pt}{0.400pt}}
\put(365,154.67){\rule{2.891pt}{0.400pt}}
\multiput(365.00,154.17)(6.000,1.000){2}{\rule{1.445pt}{0.400pt}}
\put(377,156.17){\rule{2.700pt}{0.400pt}}
\multiput(377.00,155.17)(7.396,2.000){2}{\rule{1.350pt}{0.400pt}}
\put(390,158.17){\rule{2.500pt}{0.400pt}}
\multiput(390.00,157.17)(6.811,2.000){2}{\rule{1.250pt}{0.400pt}}
\put(402,160.17){\rule{2.500pt}{0.400pt}}
\multiput(402.00,159.17)(6.811,2.000){2}{\rule{1.250pt}{0.400pt}}
\put(414,162.17){\rule{2.500pt}{0.400pt}}
\multiput(414.00,161.17)(6.811,2.000){2}{\rule{1.250pt}{0.400pt}}
\multiput(426.00,164.61)(2.472,0.447){3}{\rule{1.700pt}{0.108pt}}
\multiput(426.00,163.17)(8.472,3.000){2}{\rule{0.850pt}{0.400pt}}
\put(438,167.17){\rule{2.700pt}{0.400pt}}
\multiput(438.00,166.17)(7.396,2.000){2}{\rule{1.350pt}{0.400pt}}
\multiput(451.00,169.61)(2.472,0.447){3}{\rule{1.700pt}{0.108pt}}
\multiput(451.00,168.17)(8.472,3.000){2}{\rule{0.850pt}{0.400pt}}
\multiput(463.00,172.61)(2.472,0.447){3}{\rule{1.700pt}{0.108pt}}
\multiput(463.00,171.17)(8.472,3.000){2}{\rule{0.850pt}{0.400pt}}
\put(475,175.17){\rule{2.500pt}{0.400pt}}
\multiput(475.00,174.17)(6.811,2.000){2}{\rule{1.250pt}{0.400pt}}
\multiput(487.00,177.61)(2.472,0.447){3}{\rule{1.700pt}{0.108pt}}
\multiput(487.00,176.17)(8.472,3.000){2}{\rule{0.850pt}{0.400pt}}
\multiput(499.00,180.60)(1.797,0.468){5}{\rule{1.400pt}{0.113pt}}
\multiput(499.00,179.17)(10.094,4.000){2}{\rule{0.700pt}{0.400pt}}
\multiput(512.00,184.61)(2.472,0.447){3}{\rule{1.700pt}{0.108pt}}
\multiput(512.00,183.17)(8.472,3.000){2}{\rule{0.850pt}{0.400pt}}
\multiput(524.00,187.61)(2.472,0.447){3}{\rule{1.700pt}{0.108pt}}
\multiput(524.00,186.17)(8.472,3.000){2}{\rule{0.850pt}{0.400pt}}
\multiput(536.00,190.60)(1.651,0.468){5}{\rule{1.300pt}{0.113pt}}
\multiput(536.00,189.17)(9.302,4.000){2}{\rule{0.650pt}{0.400pt}}
\multiput(548.00,194.61)(2.472,0.447){3}{\rule{1.700pt}{0.108pt}}
\multiput(548.00,193.17)(8.472,3.000){2}{\rule{0.850pt}{0.400pt}}
\multiput(560.00,197.60)(1.797,0.468){5}{\rule{1.400pt}{0.113pt}}
\multiput(560.00,196.17)(10.094,4.000){2}{\rule{0.700pt}{0.400pt}}
\multiput(573.00,201.60)(1.651,0.468){5}{\rule{1.300pt}{0.113pt}}
\multiput(573.00,200.17)(9.302,4.000){2}{\rule{0.650pt}{0.400pt}}
\multiput(585.00,205.60)(1.651,0.468){5}{\rule{1.300pt}{0.113pt}}
\multiput(585.00,204.17)(9.302,4.000){2}{\rule{0.650pt}{0.400pt}}
\multiput(597.00,209.60)(1.651,0.468){5}{\rule{1.300pt}{0.113pt}}
\multiput(597.00,208.17)(9.302,4.000){2}{\rule{0.650pt}{0.400pt}}
\multiput(609.00,213.60)(1.651,0.468){5}{\rule{1.300pt}{0.113pt}}
\multiput(609.00,212.17)(9.302,4.000){2}{\rule{0.650pt}{0.400pt}}
\multiput(621.00,217.59)(1.378,0.477){7}{\rule{1.140pt}{0.115pt}}
\multiput(621.00,216.17)(10.634,5.000){2}{\rule{0.570pt}{0.400pt}}
\multiput(634.00,222.60)(1.651,0.468){5}{\rule{1.300pt}{0.113pt}}
\multiput(634.00,221.17)(9.302,4.000){2}{\rule{0.650pt}{0.400pt}}
\multiput(646.00,226.59)(1.267,0.477){7}{\rule{1.060pt}{0.115pt}}
\multiput(646.00,225.17)(9.800,5.000){2}{\rule{0.530pt}{0.400pt}}
\multiput(658.00,231.60)(1.651,0.468){5}{\rule{1.300pt}{0.113pt}}
\multiput(658.00,230.17)(9.302,4.000){2}{\rule{0.650pt}{0.400pt}}
\multiput(670.00,235.59)(1.267,0.477){7}{\rule{1.060pt}{0.115pt}}
\multiput(670.00,234.17)(9.800,5.000){2}{\rule{0.530pt}{0.400pt}}
\multiput(682.00,240.59)(1.378,0.477){7}{\rule{1.140pt}{0.115pt}}
\multiput(682.00,239.17)(10.634,5.000){2}{\rule{0.570pt}{0.400pt}}
\multiput(695.00,245.59)(1.267,0.477){7}{\rule{1.060pt}{0.115pt}}
\multiput(695.00,244.17)(9.800,5.000){2}{\rule{0.530pt}{0.400pt}}
\multiput(707.00,250.59)(1.267,0.477){7}{\rule{1.060pt}{0.115pt}}
\multiput(707.00,249.17)(9.800,5.000){2}{\rule{0.530pt}{0.400pt}}
\multiput(719.00,255.59)(1.267,0.477){7}{\rule{1.060pt}{0.115pt}}
\multiput(719.00,254.17)(9.800,5.000){2}{\rule{0.530pt}{0.400pt}}
\multiput(731.00,260.59)(1.033,0.482){9}{\rule{0.900pt}{0.116pt}}
\multiput(731.00,259.17)(10.132,6.000){2}{\rule{0.450pt}{0.400pt}}
\multiput(743.00,266.59)(1.378,0.477){7}{\rule{1.140pt}{0.115pt}}
\multiput(743.00,265.17)(10.634,5.000){2}{\rule{0.570pt}{0.400pt}}
\multiput(756.00,271.59)(1.267,0.477){7}{\rule{1.060pt}{0.115pt}}
\multiput(756.00,270.17)(9.800,5.000){2}{\rule{0.530pt}{0.400pt}}
\multiput(768.00,276.59)(1.033,0.482){9}{\rule{0.900pt}{0.116pt}}
\multiput(768.00,275.17)(10.132,6.000){2}{\rule{0.450pt}{0.400pt}}
\multiput(780.00,282.59)(1.267,0.477){7}{\rule{1.060pt}{0.115pt}}
\multiput(780.00,281.17)(9.800,5.000){2}{\rule{0.530pt}{0.400pt}}
\multiput(792.00,287.59)(1.033,0.482){9}{\rule{0.900pt}{0.116pt}}
\multiput(792.00,286.17)(10.132,6.000){2}{\rule{0.450pt}{0.400pt}}
\multiput(804.00,293.59)(1.123,0.482){9}{\rule{0.967pt}{0.116pt}}
\multiput(804.00,292.17)(10.994,6.000){2}{\rule{0.483pt}{0.400pt}}
\multiput(817.00,299.59)(1.033,0.482){9}{\rule{0.900pt}{0.116pt}}
\multiput(817.00,298.17)(10.132,6.000){2}{\rule{0.450pt}{0.400pt}}
\multiput(829.00,305.59)(1.033,0.482){9}{\rule{0.900pt}{0.116pt}}
\multiput(829.00,304.17)(10.132,6.000){2}{\rule{0.450pt}{0.400pt}}
\multiput(841.00,311.59)(1.033,0.482){9}{\rule{0.900pt}{0.116pt}}
\multiput(841.00,310.17)(10.132,6.000){2}{\rule{0.450pt}{0.400pt}}
\multiput(853.00,317.59)(1.123,0.482){9}{\rule{0.967pt}{0.116pt}}
\multiput(853.00,316.17)(10.994,6.000){2}{\rule{0.483pt}{0.400pt}}
\multiput(866.00,323.59)(1.033,0.482){9}{\rule{0.900pt}{0.116pt}}
\multiput(866.00,322.17)(10.132,6.000){2}{\rule{0.450pt}{0.400pt}}
\multiput(878.00,329.59)(1.033,0.482){9}{\rule{0.900pt}{0.116pt}}
\multiput(878.00,328.17)(10.132,6.000){2}{\rule{0.450pt}{0.400pt}}
\multiput(890.00,335.59)(0.874,0.485){11}{\rule{0.786pt}{0.117pt}}
\multiput(890.00,334.17)(10.369,7.000){2}{\rule{0.393pt}{0.400pt}}
\multiput(902.00,342.59)(1.033,0.482){9}{\rule{0.900pt}{0.116pt}}
\multiput(902.00,341.17)(10.132,6.000){2}{\rule{0.450pt}{0.400pt}}
\multiput(914.00,348.59)(0.950,0.485){11}{\rule{0.843pt}{0.117pt}}
\multiput(914.00,347.17)(11.251,7.000){2}{\rule{0.421pt}{0.400pt}}
\multiput(927.00,355.59)(1.033,0.482){9}{\rule{0.900pt}{0.116pt}}
\multiput(927.00,354.17)(10.132,6.000){2}{\rule{0.450pt}{0.400pt}}
\multiput(939.00,361.59)(0.874,0.485){11}{\rule{0.786pt}{0.117pt}}
\multiput(939.00,360.17)(10.369,7.000){2}{\rule{0.393pt}{0.400pt}}
\multiput(951.00,368.59)(0.874,0.485){11}{\rule{0.786pt}{0.117pt}}
\multiput(951.00,367.17)(10.369,7.000){2}{\rule{0.393pt}{0.400pt}}
\multiput(963.00,375.59)(1.033,0.482){9}{\rule{0.900pt}{0.116pt}}
\multiput(963.00,374.17)(10.132,6.000){2}{\rule{0.450pt}{0.400pt}}
\multiput(975.00,381.59)(0.950,0.485){11}{\rule{0.843pt}{0.117pt}}
\multiput(975.00,380.17)(11.251,7.000){2}{\rule{0.421pt}{0.400pt}}
\multiput(988.00,388.59)(0.874,0.485){11}{\rule{0.786pt}{0.117pt}}
\multiput(988.00,387.17)(10.369,7.000){2}{\rule{0.393pt}{0.400pt}}
\multiput(1000.00,395.59)(0.874,0.485){11}{\rule{0.786pt}{0.117pt}}
\multiput(1000.00,394.17)(10.369,7.000){2}{\rule{0.393pt}{0.400pt}}
\multiput(1012.00,402.59)(0.874,0.485){11}{\rule{0.786pt}{0.117pt}}
\multiput(1012.00,401.17)(10.369,7.000){2}{\rule{0.393pt}{0.400pt}}
\multiput(1024.00,409.59)(0.874,0.485){11}{\rule{0.786pt}{0.117pt}}
\multiput(1024.00,408.17)(10.369,7.000){2}{\rule{0.393pt}{0.400pt}}
\multiput(1036.00,416.59)(0.950,0.485){11}{\rule{0.843pt}{0.117pt}}
\multiput(1036.00,415.17)(11.251,7.000){2}{\rule{0.421pt}{0.400pt}}
\multiput(1049.00,423.59)(0.874,0.485){11}{\rule{0.786pt}{0.117pt}}
\multiput(1049.00,422.17)(10.369,7.000){2}{\rule{0.393pt}{0.400pt}}
\multiput(1061.00,430.59)(0.758,0.488){13}{\rule{0.700pt}{0.117pt}}
\multiput(1061.00,429.17)(10.547,8.000){2}{\rule{0.350pt}{0.400pt}}
\multiput(1073.00,438.59)(0.874,0.485){11}{\rule{0.786pt}{0.117pt}}
\multiput(1073.00,437.17)(10.369,7.000){2}{\rule{0.393pt}{0.400pt}}
\multiput(1085.00,445.59)(0.874,0.485){11}{\rule{0.786pt}{0.117pt}}
\multiput(1085.00,444.17)(10.369,7.000){2}{\rule{0.393pt}{0.400pt}}
\multiput(1097.00,452.59)(0.824,0.488){13}{\rule{0.750pt}{0.117pt}}
\multiput(1097.00,451.17)(11.443,8.000){2}{\rule{0.375pt}{0.400pt}}
\multiput(1110.00,460.59)(0.874,0.485){11}{\rule{0.786pt}{0.117pt}}
\multiput(1110.00,459.17)(10.369,7.000){2}{\rule{0.393pt}{0.400pt}}
\multiput(1122.00,467.59)(0.874,0.485){11}{\rule{0.786pt}{0.117pt}}
\multiput(1122.00,466.17)(10.369,7.000){2}{\rule{0.393pt}{0.400pt}}
\multiput(1134.00,474.59)(0.758,0.488){13}{\rule{0.700pt}{0.117pt}}
\multiput(1134.00,473.17)(10.547,8.000){2}{\rule{0.350pt}{0.400pt}}
\multiput(1146.00,482.59)(0.874,0.485){11}{\rule{0.786pt}{0.117pt}}
\multiput(1146.00,481.17)(10.369,7.000){2}{\rule{0.393pt}{0.400pt}}
\multiput(1158.00,489.59)(0.824,0.488){13}{\rule{0.750pt}{0.117pt}}
\multiput(1158.00,488.17)(11.443,8.000){2}{\rule{0.375pt}{0.400pt}}
\multiput(1171.00,497.59)(0.758,0.488){13}{\rule{0.700pt}{0.117pt}}
\multiput(1171.00,496.17)(10.547,8.000){2}{\rule{0.350pt}{0.400pt}}
\multiput(1183.00,505.59)(0.874,0.485){11}{\rule{0.786pt}{0.117pt}}
\multiput(1183.00,504.17)(10.369,7.000){2}{\rule{0.393pt}{0.400pt}}
\multiput(1195.00,512.59)(0.758,0.488){13}{\rule{0.700pt}{0.117pt}}
\multiput(1195.00,511.17)(10.547,8.000){2}{\rule{0.350pt}{0.400pt}}
\multiput(1207.00,520.59)(0.758,0.488){13}{\rule{0.700pt}{0.117pt}}
\multiput(1207.00,519.17)(10.547,8.000){2}{\rule{0.350pt}{0.400pt}}
\multiput(1219.00,528.59)(0.950,0.485){11}{\rule{0.843pt}{0.117pt}}
\multiput(1219.00,527.17)(11.251,7.000){2}{\rule{0.421pt}{0.400pt}}
\multiput(1232.00,535.59)(0.758,0.488){13}{\rule{0.700pt}{0.117pt}}
\multiput(1232.00,534.17)(10.547,8.000){2}{\rule{0.350pt}{0.400pt}}
\multiput(1244.00,543.59)(0.758,0.488){13}{\rule{0.700pt}{0.117pt}}
\multiput(1244.00,542.17)(10.547,8.000){2}{\rule{0.350pt}{0.400pt}}
\multiput(1256.00,551.59)(0.758,0.488){13}{\rule{0.700pt}{0.117pt}}
\multiput(1256.00,550.17)(10.547,8.000){2}{\rule{0.350pt}{0.400pt}}
\multiput(1268.00,559.59)(0.758,0.488){13}{\rule{0.700pt}{0.117pt}}
\multiput(1268.00,558.17)(10.547,8.000){2}{\rule{0.350pt}{0.400pt}}
\multiput(1280.00,567.59)(0.950,0.485){11}{\rule{0.843pt}{0.117pt}}
\multiput(1280.00,566.17)(11.251,7.000){2}{\rule{0.421pt}{0.400pt}}
\multiput(1293.00,574.59)(0.758,0.488){13}{\rule{0.700pt}{0.117pt}}
\multiput(1293.00,573.17)(10.547,8.000){2}{\rule{0.350pt}{0.400pt}}
\multiput(1305.00,582.59)(0.758,0.488){13}{\rule{0.700pt}{0.117pt}}
\multiput(1305.00,581.17)(10.547,8.000){2}{\rule{0.350pt}{0.400pt}}
\multiput(1317.00,590.59)(0.758,0.488){13}{\rule{0.700pt}{0.117pt}}
\multiput(1317.00,589.17)(10.547,8.000){2}{\rule{0.350pt}{0.400pt}}
\multiput(1329.00,598.59)(0.758,0.488){13}{\rule{0.700pt}{0.117pt}}
\multiput(1329.00,597.17)(10.547,8.000){2}{\rule{0.350pt}{0.400pt}}
\multiput(1341.00,606.59)(0.824,0.488){13}{\rule{0.750pt}{0.117pt}}
\multiput(1341.00,605.17)(11.443,8.000){2}{\rule{0.375pt}{0.400pt}}
\multiput(1354.00,614.59)(0.758,0.488){13}{\rule{0.700pt}{0.117pt}}
\multiput(1354.00,613.17)(10.547,8.000){2}{\rule{0.350pt}{0.400pt}}
\multiput(1366.00,622.59)(0.758,0.488){13}{\rule{0.700pt}{0.117pt}}
\multiput(1366.00,621.17)(10.547,8.000){2}{\rule{0.350pt}{0.400pt}}
\multiput(1378.00,630.59)(0.758,0.488){13}{\rule{0.700pt}{0.117pt}}
\multiput(1378.00,629.17)(10.547,8.000){2}{\rule{0.350pt}{0.400pt}}
\multiput(1390.00,638.59)(0.758,0.488){13}{\rule{0.700pt}{0.117pt}}
\multiput(1390.00,637.17)(10.547,8.000){2}{\rule{0.350pt}{0.400pt}}
\multiput(1402.00,646.59)(0.824,0.488){13}{\rule{0.750pt}{0.117pt}}
\multiput(1402.00,645.17)(11.443,8.000){2}{\rule{0.375pt}{0.400pt}}
\multiput(1415.00,654.59)(0.758,0.488){13}{\rule{0.700pt}{0.117pt}}
\multiput(1415.00,653.17)(10.547,8.000){2}{\rule{0.350pt}{0.400pt}}
\multiput(1427.00,662.59)(0.758,0.488){13}{\rule{0.700pt}{0.117pt}}
\multiput(1427.00,661.17)(10.547,8.000){2}{\rule{0.350pt}{0.400pt}}
\put(280.0,147.0){\rule[-0.200pt]{2.891pt}{0.400pt}}
\put(1279,654){\makebox(0,0)[r]{Algorytm 2}}
\multiput(1299,654)(20.756,0.000){5}{\usebox{\plotpoint}}
\put(1399,654){\usebox{\plotpoint}}
\put(231,132){\usebox{\plotpoint}}
\put(231.00,132.00){\usebox{\plotpoint}}
\put(251.76,132.00){\usebox{\plotpoint}}
\put(272.51,132.00){\usebox{\plotpoint}}
\put(293.27,132.00){\usebox{\plotpoint}}
\put(314.02,132.00){\usebox{\plotpoint}}
\put(334.78,132.00){\usebox{\plotpoint}}
\put(355.53,132.00){\usebox{\plotpoint}}
\put(376.29,132.00){\usebox{\plotpoint}}
\put(397.01,133.00){\usebox{\plotpoint}}
\put(417.76,133.00){\usebox{\plotpoint}}
\put(438.52,133.00){\usebox{\plotpoint}}
\put(459.27,133.00){\usebox{\plotpoint}}
\put(480.03,133.00){\usebox{\plotpoint}}
\put(500.78,133.00){\usebox{\plotpoint}}
\put(521.54,133.00){\usebox{\plotpoint}}
\put(542.29,133.00){\usebox{\plotpoint}}
\put(563.05,133.00){\usebox{\plotpoint}}
\put(583.80,133.00){\usebox{\plotpoint}}
\put(604.53,133.63){\usebox{\plotpoint}}
\put(625.27,134.00){\usebox{\plotpoint}}
\put(646.03,134.00){\usebox{\plotpoint}}
\put(666.79,134.00){\usebox{\plotpoint}}
\put(687.54,134.00){\usebox{\plotpoint}}
\put(708.30,134.00){\usebox{\plotpoint}}
\put(729.05,134.00){\usebox{\plotpoint}}
\put(749.81,134.00){\usebox{\plotpoint}}
\put(770.56,134.00){\usebox{\plotpoint}}
\put(791.32,134.00){\usebox{\plotpoint}}
\put(812.07,134.00){\usebox{\plotpoint}}
\put(832.83,134.00){\usebox{\plotpoint}}
\put(853.58,134.00){\usebox{\plotpoint}}
\put(874.34,134.00){\usebox{\plotpoint}}
\put(895.05,135.00){\usebox{\plotpoint}}
\put(915.81,135.00){\usebox{\plotpoint}}
\put(936.57,135.00){\usebox{\plotpoint}}
\put(957.32,135.00){\usebox{\plotpoint}}
\put(978.08,135.00){\usebox{\plotpoint}}
\put(998.83,135.00){\usebox{\plotpoint}}
\put(1019.59,135.00){\usebox{\plotpoint}}
\put(1040.34,135.00){\usebox{\plotpoint}}
\put(1061.10,135.00){\usebox{\plotpoint}}
\put(1081.85,135.00){\usebox{\plotpoint}}
\put(1102.61,135.00){\usebox{\plotpoint}}
\put(1123.36,135.00){\usebox{\plotpoint}}
\put(1144.12,135.00){\usebox{\plotpoint}}
\put(1164.88,135.00){\usebox{\plotpoint}}
\put(1185.63,135.00){\usebox{\plotpoint}}
\put(1206.39,135.00){\usebox{\plotpoint}}
\put(1227.14,135.00){\usebox{\plotpoint}}
\put(1247.90,135.00){\usebox{\plotpoint}}
\put(1268.65,135.00){\usebox{\plotpoint}}
\put(1289.38,135.72){\usebox{\plotpoint}}
\put(1310.13,136.00){\usebox{\plotpoint}}
\put(1330.88,136.00){\usebox{\plotpoint}}
\put(1351.64,136.00){\usebox{\plotpoint}}
\put(1372.39,136.00){\usebox{\plotpoint}}
\put(1393.15,136.00){\usebox{\plotpoint}}
\put(1413.90,136.00){\usebox{\plotpoint}}
\put(1434.66,136.00){\usebox{\plotpoint}}
\put(1439,136){\usebox{\plotpoint}}
\put(231.0,131.0){\rule[-0.200pt]{0.400pt}{155.380pt}}
\put(231.0,131.0){\rule[-0.200pt]{291.007pt}{0.400pt}}
\put(1439.0,131.0){\rule[-0.200pt]{0.400pt}{155.380pt}}
\put(231.0,776.0){\rule[-0.200pt]{291.007pt}{0.400pt}}
\end{picture}

\caption{Wykres wydajnosci}

% GNUPLOT: LaTeX picture
\setlength{\unitlength}{0.240900pt}
\ifx\plotpoint\undefined\newsavebox{\plotpoint}\fi
\begin{picture}(1500,900)(0,0)
\sbox{\plotpoint}{\rule[-0.200pt]{0.400pt}{0.400pt}}%
\put(231.0,131.0){\rule[-0.200pt]{291.007pt}{0.400pt}}
\put(231.0,131.0){\rule[-0.200pt]{4.818pt}{0.400pt}}
\put(211,131){\makebox(0,0)[r]{ 0}}
\put(1419.0,131.0){\rule[-0.200pt]{4.818pt}{0.400pt}}
\put(231.0,203.0){\rule[-0.200pt]{291.007pt}{0.400pt}}
\put(231.0,203.0){\rule[-0.200pt]{4.818pt}{0.400pt}}
\put(211,203){\makebox(0,0)[r]{ 20000}}
\put(1419.0,203.0){\rule[-0.200pt]{4.818pt}{0.400pt}}
\put(231.0,274.0){\rule[-0.200pt]{291.007pt}{0.400pt}}
\put(231.0,274.0){\rule[-0.200pt]{4.818pt}{0.400pt}}
\put(211,274){\makebox(0,0)[r]{ 40000}}
\put(1419.0,274.0){\rule[-0.200pt]{4.818pt}{0.400pt}}
\put(231.0,346.0){\rule[-0.200pt]{291.007pt}{0.400pt}}
\put(231.0,346.0){\rule[-0.200pt]{4.818pt}{0.400pt}}
\put(211,346){\makebox(0,0)[r]{ 60000}}
\put(1419.0,346.0){\rule[-0.200pt]{4.818pt}{0.400pt}}
\put(231.0,418.0){\rule[-0.200pt]{291.007pt}{0.400pt}}
\put(231.0,418.0){\rule[-0.200pt]{4.818pt}{0.400pt}}
\put(211,418){\makebox(0,0)[r]{ 80000}}
\put(1419.0,418.0){\rule[-0.200pt]{4.818pt}{0.400pt}}
\put(231.0,489.0){\rule[-0.200pt]{291.007pt}{0.400pt}}
\put(231.0,489.0){\rule[-0.200pt]{4.818pt}{0.400pt}}
\put(211,489){\makebox(0,0)[r]{ 100000}}
\put(1419.0,489.0){\rule[-0.200pt]{4.818pt}{0.400pt}}
\put(231.0,561.0){\rule[-0.200pt]{291.007pt}{0.400pt}}
\put(231.0,561.0){\rule[-0.200pt]{4.818pt}{0.400pt}}
\put(211,561){\makebox(0,0)[r]{ 120000}}
\put(1419.0,561.0){\rule[-0.200pt]{4.818pt}{0.400pt}}
\put(231.0,633.0){\rule[-0.200pt]{291.007pt}{0.400pt}}
\put(231.0,633.0){\rule[-0.200pt]{4.818pt}{0.400pt}}
\put(211,633){\makebox(0,0)[r]{ 140000}}
\put(1419.0,633.0){\rule[-0.200pt]{4.818pt}{0.400pt}}
\put(231.0,704.0){\rule[-0.200pt]{204.283pt}{0.400pt}}
\put(1419.0,704.0){\rule[-0.200pt]{4.818pt}{0.400pt}}
\put(231.0,704.0){\rule[-0.200pt]{4.818pt}{0.400pt}}
\put(211,704){\makebox(0,0)[r]{ 160000}}
\put(1419.0,704.0){\rule[-0.200pt]{4.818pt}{0.400pt}}
\put(231.0,776.0){\rule[-0.200pt]{291.007pt}{0.400pt}}
\put(231.0,776.0){\rule[-0.200pt]{4.818pt}{0.400pt}}
\put(211,776){\makebox(0,0)[r]{ 180000}}
\put(1419.0,776.0){\rule[-0.200pt]{4.818pt}{0.400pt}}
\put(295.0,131.0){\rule[-0.200pt]{0.400pt}{155.380pt}}
\put(295.0,131.0){\rule[-0.200pt]{0.400pt}{4.818pt}}
\put(295,90){\makebox(0,0){ 20}}
\put(295.0,756.0){\rule[-0.200pt]{0.400pt}{4.818pt}}
\put(422.0,131.0){\rule[-0.200pt]{0.400pt}{155.380pt}}
\put(422.0,131.0){\rule[-0.200pt]{0.400pt}{4.818pt}}
\put(422,90){\makebox(0,0){ 40}}
\put(422.0,756.0){\rule[-0.200pt]{0.400pt}{4.818pt}}
\put(549.0,131.0){\rule[-0.200pt]{0.400pt}{155.380pt}}
\put(549.0,131.0){\rule[-0.200pt]{0.400pt}{4.818pt}}
\put(549,90){\makebox(0,0){ 60}}
\put(549.0,756.0){\rule[-0.200pt]{0.400pt}{4.818pt}}
\put(676.0,131.0){\rule[-0.200pt]{0.400pt}{155.380pt}}
\put(676.0,131.0){\rule[-0.200pt]{0.400pt}{4.818pt}}
\put(676,90){\makebox(0,0){ 80}}
\put(676.0,756.0){\rule[-0.200pt]{0.400pt}{4.818pt}}
\put(803.0,131.0){\rule[-0.200pt]{0.400pt}{155.380pt}}
\put(803.0,131.0){\rule[-0.200pt]{0.400pt}{4.818pt}}
\put(803,90){\makebox(0,0){ 100}}
\put(803.0,756.0){\rule[-0.200pt]{0.400pt}{4.818pt}}
\put(930.0,131.0){\rule[-0.200pt]{0.400pt}{155.380pt}}
\put(930.0,131.0){\rule[-0.200pt]{0.400pt}{4.818pt}}
\put(930,90){\makebox(0,0){ 120}}
\put(930.0,756.0){\rule[-0.200pt]{0.400pt}{4.818pt}}
\put(1058.0,131.0){\rule[-0.200pt]{0.400pt}{155.380pt}}
\put(1058.0,131.0){\rule[-0.200pt]{0.400pt}{4.818pt}}
\put(1058,90){\makebox(0,0){ 140}}
\put(1058.0,756.0){\rule[-0.200pt]{0.400pt}{4.818pt}}
\put(1185.0,131.0){\rule[-0.200pt]{0.400pt}{120.932pt}}
\put(1185.0,756.0){\rule[-0.200pt]{0.400pt}{4.818pt}}
\put(1185.0,131.0){\rule[-0.200pt]{0.400pt}{4.818pt}}
\put(1185,90){\makebox(0,0){ 160}}
\put(1185.0,756.0){\rule[-0.200pt]{0.400pt}{4.818pt}}
\put(1312.0,131.0){\rule[-0.200pt]{0.400pt}{120.932pt}}
\put(1312.0,756.0){\rule[-0.200pt]{0.400pt}{4.818pt}}
\put(1312.0,131.0){\rule[-0.200pt]{0.400pt}{4.818pt}}
\put(1312,90){\makebox(0,0){ 180}}
\put(1312.0,756.0){\rule[-0.200pt]{0.400pt}{4.818pt}}
\put(1439.0,131.0){\rule[-0.200pt]{0.400pt}{155.380pt}}
\put(1439.0,131.0){\rule[-0.200pt]{0.400pt}{4.818pt}}
\put(1439,90){\makebox(0,0){ 200}}
\put(1439.0,756.0){\rule[-0.200pt]{0.400pt}{4.818pt}}
\put(231.0,131.0){\rule[-0.200pt]{0.400pt}{155.380pt}}
\put(231.0,131.0){\rule[-0.200pt]{291.007pt}{0.400pt}}
\put(1439.0,131.0){\rule[-0.200pt]{0.400pt}{155.380pt}}
\put(231.0,776.0){\rule[-0.200pt]{291.007pt}{0.400pt}}
\put(30,453){\makebox(0,0){Czas w [s]}}
\put(835,29){\makebox(0,0){Wielkosc problemy (ilosc danych)}}
\put(835,838){\makebox(0,0){Wykres wydajnosci algorytmow}}
\put(1249,735){\makebox(0,0){Legenda}}
\put(1279,695){\makebox(0,0)[r]{Algorytm 1}}
\put(1299.0,695.0){\rule[-0.200pt]{24.090pt}{0.400pt}}
\put(231,141){\usebox{\plotpoint}}
\put(231,140.67){\rule{2.891pt}{0.400pt}}
\multiput(231.00,140.17)(6.000,1.000){2}{\rule{1.445pt}{0.400pt}}
\put(243,142.17){\rule{2.500pt}{0.400pt}}
\multiput(243.00,141.17)(6.811,2.000){2}{\rule{1.250pt}{0.400pt}}
\put(255,144.17){\rule{2.700pt}{0.400pt}}
\multiput(255.00,143.17)(7.396,2.000){2}{\rule{1.350pt}{0.400pt}}
\put(268,146.17){\rule{2.500pt}{0.400pt}}
\multiput(268.00,145.17)(6.811,2.000){2}{\rule{1.250pt}{0.400pt}}
\put(280,147.67){\rule{2.891pt}{0.400pt}}
\multiput(280.00,147.17)(6.000,1.000){2}{\rule{1.445pt}{0.400pt}}
\put(292,149.17){\rule{2.500pt}{0.400pt}}
\multiput(292.00,148.17)(6.811,2.000){2}{\rule{1.250pt}{0.400pt}}
\put(304,151.17){\rule{2.500pt}{0.400pt}}
\multiput(304.00,150.17)(6.811,2.000){2}{\rule{1.250pt}{0.400pt}}
\put(316,153.17){\rule{2.700pt}{0.400pt}}
\multiput(316.00,152.17)(7.396,2.000){2}{\rule{1.350pt}{0.400pt}}
\put(329,154.67){\rule{2.891pt}{0.400pt}}
\multiput(329.00,154.17)(6.000,1.000){2}{\rule{1.445pt}{0.400pt}}
\put(341,156.17){\rule{2.500pt}{0.400pt}}
\multiput(341.00,155.17)(6.811,2.000){2}{\rule{1.250pt}{0.400pt}}
\put(353,158.17){\rule{2.500pt}{0.400pt}}
\multiput(353.00,157.17)(6.811,2.000){2}{\rule{1.250pt}{0.400pt}}
\put(365,160.17){\rule{2.500pt}{0.400pt}}
\multiput(365.00,159.17)(6.811,2.000){2}{\rule{1.250pt}{0.400pt}}
\put(377,162.17){\rule{2.700pt}{0.400pt}}
\multiput(377.00,161.17)(7.396,2.000){2}{\rule{1.350pt}{0.400pt}}
\put(390,164.17){\rule{2.500pt}{0.400pt}}
\multiput(390.00,163.17)(6.811,2.000){2}{\rule{1.250pt}{0.400pt}}
\put(402,165.67){\rule{2.891pt}{0.400pt}}
\multiput(402.00,165.17)(6.000,1.000){2}{\rule{1.445pt}{0.400pt}}
\put(414,167.17){\rule{2.500pt}{0.400pt}}
\multiput(414.00,166.17)(6.811,2.000){2}{\rule{1.250pt}{0.400pt}}
\put(426,169.17){\rule{2.500pt}{0.400pt}}
\multiput(426.00,168.17)(6.811,2.000){2}{\rule{1.250pt}{0.400pt}}
\put(438,171.17){\rule{2.700pt}{0.400pt}}
\multiput(438.00,170.17)(7.396,2.000){2}{\rule{1.350pt}{0.400pt}}
\put(451,173.17){\rule{2.500pt}{0.400pt}}
\multiput(451.00,172.17)(6.811,2.000){2}{\rule{1.250pt}{0.400pt}}
\put(463,175.17){\rule{2.500pt}{0.400pt}}
\multiput(463.00,174.17)(6.811,2.000){2}{\rule{1.250pt}{0.400pt}}
\put(475,177.17){\rule{2.500pt}{0.400pt}}
\multiput(475.00,176.17)(6.811,2.000){2}{\rule{1.250pt}{0.400pt}}
\put(487,179.17){\rule{2.500pt}{0.400pt}}
\multiput(487.00,178.17)(6.811,2.000){2}{\rule{1.250pt}{0.400pt}}
\put(499,181.17){\rule{2.700pt}{0.400pt}}
\multiput(499.00,180.17)(7.396,2.000){2}{\rule{1.350pt}{0.400pt}}
\put(512,183.17){\rule{2.500pt}{0.400pt}}
\multiput(512.00,182.17)(6.811,2.000){2}{\rule{1.250pt}{0.400pt}}
\put(524,185.17){\rule{2.500pt}{0.400pt}}
\multiput(524.00,184.17)(6.811,2.000){2}{\rule{1.250pt}{0.400pt}}
\put(536,187.17){\rule{2.500pt}{0.400pt}}
\multiput(536.00,186.17)(6.811,2.000){2}{\rule{1.250pt}{0.400pt}}
\multiput(548.00,189.61)(2.472,0.447){3}{\rule{1.700pt}{0.108pt}}
\multiput(548.00,188.17)(8.472,3.000){2}{\rule{0.850pt}{0.400pt}}
\put(560,192.17){\rule{2.700pt}{0.400pt}}
\multiput(560.00,191.17)(7.396,2.000){2}{\rule{1.350pt}{0.400pt}}
\put(573,194.17){\rule{2.500pt}{0.400pt}}
\multiput(573.00,193.17)(6.811,2.000){2}{\rule{1.250pt}{0.400pt}}
\put(585,196.17){\rule{2.500pt}{0.400pt}}
\multiput(585.00,195.17)(6.811,2.000){2}{\rule{1.250pt}{0.400pt}}
\multiput(597.00,198.61)(2.472,0.447){3}{\rule{1.700pt}{0.108pt}}
\multiput(597.00,197.17)(8.472,3.000){2}{\rule{0.850pt}{0.400pt}}
\put(609,201.17){\rule{2.500pt}{0.400pt}}
\multiput(609.00,200.17)(6.811,2.000){2}{\rule{1.250pt}{0.400pt}}
\put(621,203.17){\rule{2.700pt}{0.400pt}}
\multiput(621.00,202.17)(7.396,2.000){2}{\rule{1.350pt}{0.400pt}}
\multiput(634.00,205.61)(2.472,0.447){3}{\rule{1.700pt}{0.108pt}}
\multiput(634.00,204.17)(8.472,3.000){2}{\rule{0.850pt}{0.400pt}}
\put(646,208.17){\rule{2.500pt}{0.400pt}}
\multiput(646.00,207.17)(6.811,2.000){2}{\rule{1.250pt}{0.400pt}}
\multiput(658.00,210.61)(2.472,0.447){3}{\rule{1.700pt}{0.108pt}}
\multiput(658.00,209.17)(8.472,3.000){2}{\rule{0.850pt}{0.400pt}}
\put(670,213.17){\rule{2.500pt}{0.400pt}}
\multiput(670.00,212.17)(6.811,2.000){2}{\rule{1.250pt}{0.400pt}}
\multiput(682.00,215.61)(2.695,0.447){3}{\rule{1.833pt}{0.108pt}}
\multiput(682.00,214.17)(9.195,3.000){2}{\rule{0.917pt}{0.400pt}}
\multiput(695.00,218.61)(2.472,0.447){3}{\rule{1.700pt}{0.108pt}}
\multiput(695.00,217.17)(8.472,3.000){2}{\rule{0.850pt}{0.400pt}}
\multiput(707.00,221.61)(2.472,0.447){3}{\rule{1.700pt}{0.108pt}}
\multiput(707.00,220.17)(8.472,3.000){2}{\rule{0.850pt}{0.400pt}}
\put(719,224.17){\rule{2.500pt}{0.400pt}}
\multiput(719.00,223.17)(6.811,2.000){2}{\rule{1.250pt}{0.400pt}}
\multiput(731.00,226.61)(2.472,0.447){3}{\rule{1.700pt}{0.108pt}}
\multiput(731.00,225.17)(8.472,3.000){2}{\rule{0.850pt}{0.400pt}}
\multiput(743.00,229.61)(2.695,0.447){3}{\rule{1.833pt}{0.108pt}}
\multiput(743.00,228.17)(9.195,3.000){2}{\rule{0.917pt}{0.400pt}}
\multiput(756.00,232.61)(2.472,0.447){3}{\rule{1.700pt}{0.108pt}}
\multiput(756.00,231.17)(8.472,3.000){2}{\rule{0.850pt}{0.400pt}}
\multiput(768.00,235.61)(2.472,0.447){3}{\rule{1.700pt}{0.108pt}}
\multiput(768.00,234.17)(8.472,3.000){2}{\rule{0.850pt}{0.400pt}}
\multiput(780.00,238.61)(2.472,0.447){3}{\rule{1.700pt}{0.108pt}}
\multiput(780.00,237.17)(8.472,3.000){2}{\rule{0.850pt}{0.400pt}}
\multiput(792.00,241.61)(2.472,0.447){3}{\rule{1.700pt}{0.108pt}}
\multiput(792.00,240.17)(8.472,3.000){2}{\rule{0.850pt}{0.400pt}}
\multiput(804.00,244.61)(2.695,0.447){3}{\rule{1.833pt}{0.108pt}}
\multiput(804.00,243.17)(9.195,3.000){2}{\rule{0.917pt}{0.400pt}}
\multiput(817.00,247.61)(2.472,0.447){3}{\rule{1.700pt}{0.108pt}}
\multiput(817.00,246.17)(8.472,3.000){2}{\rule{0.850pt}{0.400pt}}
\multiput(829.00,250.60)(1.651,0.468){5}{\rule{1.300pt}{0.113pt}}
\multiput(829.00,249.17)(9.302,4.000){2}{\rule{0.650pt}{0.400pt}}
\multiput(841.00,254.61)(2.472,0.447){3}{\rule{1.700pt}{0.108pt}}
\multiput(841.00,253.17)(8.472,3.000){2}{\rule{0.850pt}{0.400pt}}
\multiput(853.00,257.61)(2.695,0.447){3}{\rule{1.833pt}{0.108pt}}
\multiput(853.00,256.17)(9.195,3.000){2}{\rule{0.917pt}{0.400pt}}
\multiput(866.00,260.60)(1.651,0.468){5}{\rule{1.300pt}{0.113pt}}
\multiput(866.00,259.17)(9.302,4.000){2}{\rule{0.650pt}{0.400pt}}
\multiput(878.00,264.61)(2.472,0.447){3}{\rule{1.700pt}{0.108pt}}
\multiput(878.00,263.17)(8.472,3.000){2}{\rule{0.850pt}{0.400pt}}
\multiput(890.00,267.60)(1.651,0.468){5}{\rule{1.300pt}{0.113pt}}
\multiput(890.00,266.17)(9.302,4.000){2}{\rule{0.650pt}{0.400pt}}
\multiput(902.00,271.60)(1.651,0.468){5}{\rule{1.300pt}{0.113pt}}
\multiput(902.00,270.17)(9.302,4.000){2}{\rule{0.650pt}{0.400pt}}
\multiput(914.00,275.61)(2.695,0.447){3}{\rule{1.833pt}{0.108pt}}
\multiput(914.00,274.17)(9.195,3.000){2}{\rule{0.917pt}{0.400pt}}
\multiput(927.00,278.60)(1.651,0.468){5}{\rule{1.300pt}{0.113pt}}
\multiput(927.00,277.17)(9.302,4.000){2}{\rule{0.650pt}{0.400pt}}
\multiput(939.00,282.60)(1.651,0.468){5}{\rule{1.300pt}{0.113pt}}
\multiput(939.00,281.17)(9.302,4.000){2}{\rule{0.650pt}{0.400pt}}
\multiput(951.00,286.60)(1.651,0.468){5}{\rule{1.300pt}{0.113pt}}
\multiput(951.00,285.17)(9.302,4.000){2}{\rule{0.650pt}{0.400pt}}
\multiput(963.00,290.60)(1.651,0.468){5}{\rule{1.300pt}{0.113pt}}
\multiput(963.00,289.17)(9.302,4.000){2}{\rule{0.650pt}{0.400pt}}
\multiput(975.00,294.60)(1.797,0.468){5}{\rule{1.400pt}{0.113pt}}
\multiput(975.00,293.17)(10.094,4.000){2}{\rule{0.700pt}{0.400pt}}
\multiput(988.00,298.60)(1.651,0.468){5}{\rule{1.300pt}{0.113pt}}
\multiput(988.00,297.17)(9.302,4.000){2}{\rule{0.650pt}{0.400pt}}
\multiput(1000.00,302.60)(1.651,0.468){5}{\rule{1.300pt}{0.113pt}}
\multiput(1000.00,301.17)(9.302,4.000){2}{\rule{0.650pt}{0.400pt}}
\multiput(1012.00,306.60)(1.651,0.468){5}{\rule{1.300pt}{0.113pt}}
\multiput(1012.00,305.17)(9.302,4.000){2}{\rule{0.650pt}{0.400pt}}
\multiput(1024.00,310.60)(1.651,0.468){5}{\rule{1.300pt}{0.113pt}}
\multiput(1024.00,309.17)(9.302,4.000){2}{\rule{0.650pt}{0.400pt}}
\multiput(1036.00,314.59)(1.378,0.477){7}{\rule{1.140pt}{0.115pt}}
\multiput(1036.00,313.17)(10.634,5.000){2}{\rule{0.570pt}{0.400pt}}
\multiput(1049.00,319.60)(1.651,0.468){5}{\rule{1.300pt}{0.113pt}}
\multiput(1049.00,318.17)(9.302,4.000){2}{\rule{0.650pt}{0.400pt}}
\multiput(1061.00,323.60)(1.651,0.468){5}{\rule{1.300pt}{0.113pt}}
\multiput(1061.00,322.17)(9.302,4.000){2}{\rule{0.650pt}{0.400pt}}
\multiput(1073.00,327.59)(1.267,0.477){7}{\rule{1.060pt}{0.115pt}}
\multiput(1073.00,326.17)(9.800,5.000){2}{\rule{0.530pt}{0.400pt}}
\multiput(1085.00,332.60)(1.651,0.468){5}{\rule{1.300pt}{0.113pt}}
\multiput(1085.00,331.17)(9.302,4.000){2}{\rule{0.650pt}{0.400pt}}
\multiput(1097.00,336.59)(1.378,0.477){7}{\rule{1.140pt}{0.115pt}}
\multiput(1097.00,335.17)(10.634,5.000){2}{\rule{0.570pt}{0.400pt}}
\multiput(1110.00,341.60)(1.651,0.468){5}{\rule{1.300pt}{0.113pt}}
\multiput(1110.00,340.17)(9.302,4.000){2}{\rule{0.650pt}{0.400pt}}
\multiput(1122.00,345.59)(1.267,0.477){7}{\rule{1.060pt}{0.115pt}}
\multiput(1122.00,344.17)(9.800,5.000){2}{\rule{0.530pt}{0.400pt}}
\multiput(1134.00,350.59)(1.267,0.477){7}{\rule{1.060pt}{0.115pt}}
\multiput(1134.00,349.17)(9.800,5.000){2}{\rule{0.530pt}{0.400pt}}
\multiput(1146.00,355.59)(1.267,0.477){7}{\rule{1.060pt}{0.115pt}}
\multiput(1146.00,354.17)(9.800,5.000){2}{\rule{0.530pt}{0.400pt}}
\multiput(1158.00,360.59)(1.378,0.477){7}{\rule{1.140pt}{0.115pt}}
\multiput(1158.00,359.17)(10.634,5.000){2}{\rule{0.570pt}{0.400pt}}
\multiput(1171.00,365.60)(1.651,0.468){5}{\rule{1.300pt}{0.113pt}}
\multiput(1171.00,364.17)(9.302,4.000){2}{\rule{0.650pt}{0.400pt}}
\multiput(1183.00,369.59)(1.267,0.477){7}{\rule{1.060pt}{0.115pt}}
\multiput(1183.00,368.17)(9.800,5.000){2}{\rule{0.530pt}{0.400pt}}
\multiput(1195.00,374.59)(1.267,0.477){7}{\rule{1.060pt}{0.115pt}}
\multiput(1195.00,373.17)(9.800,5.000){2}{\rule{0.530pt}{0.400pt}}
\multiput(1207.00,379.59)(1.033,0.482){9}{\rule{0.900pt}{0.116pt}}
\multiput(1207.00,378.17)(10.132,6.000){2}{\rule{0.450pt}{0.400pt}}
\multiput(1219.00,385.59)(1.378,0.477){7}{\rule{1.140pt}{0.115pt}}
\multiput(1219.00,384.17)(10.634,5.000){2}{\rule{0.570pt}{0.400pt}}
\multiput(1232.00,390.59)(1.267,0.477){7}{\rule{1.060pt}{0.115pt}}
\multiput(1232.00,389.17)(9.800,5.000){2}{\rule{0.530pt}{0.400pt}}
\multiput(1244.00,395.59)(1.267,0.477){7}{\rule{1.060pt}{0.115pt}}
\multiput(1244.00,394.17)(9.800,5.000){2}{\rule{0.530pt}{0.400pt}}
\multiput(1256.00,400.59)(1.267,0.477){7}{\rule{1.060pt}{0.115pt}}
\multiput(1256.00,399.17)(9.800,5.000){2}{\rule{0.530pt}{0.400pt}}
\multiput(1268.00,405.59)(1.033,0.482){9}{\rule{0.900pt}{0.116pt}}
\multiput(1268.00,404.17)(10.132,6.000){2}{\rule{0.450pt}{0.400pt}}
\multiput(1280.00,411.59)(1.378,0.477){7}{\rule{1.140pt}{0.115pt}}
\multiput(1280.00,410.17)(10.634,5.000){2}{\rule{0.570pt}{0.400pt}}
\multiput(1293.00,416.59)(1.033,0.482){9}{\rule{0.900pt}{0.116pt}}
\multiput(1293.00,415.17)(10.132,6.000){2}{\rule{0.450pt}{0.400pt}}
\multiput(1305.00,422.59)(1.267,0.477){7}{\rule{1.060pt}{0.115pt}}
\multiput(1305.00,421.17)(9.800,5.000){2}{\rule{0.530pt}{0.400pt}}
\multiput(1317.00,427.59)(1.033,0.482){9}{\rule{0.900pt}{0.116pt}}
\multiput(1317.00,426.17)(10.132,6.000){2}{\rule{0.450pt}{0.400pt}}
\multiput(1329.00,433.59)(1.033,0.482){9}{\rule{0.900pt}{0.116pt}}
\multiput(1329.00,432.17)(10.132,6.000){2}{\rule{0.450pt}{0.400pt}}
\multiput(1341.00,439.59)(1.378,0.477){7}{\rule{1.140pt}{0.115pt}}
\multiput(1341.00,438.17)(10.634,5.000){2}{\rule{0.570pt}{0.400pt}}
\multiput(1354.00,444.59)(1.033,0.482){9}{\rule{0.900pt}{0.116pt}}
\multiput(1354.00,443.17)(10.132,6.000){2}{\rule{0.450pt}{0.400pt}}
\multiput(1366.00,450.59)(1.033,0.482){9}{\rule{0.900pt}{0.116pt}}
\multiput(1366.00,449.17)(10.132,6.000){2}{\rule{0.450pt}{0.400pt}}
\multiput(1378.00,456.59)(1.033,0.482){9}{\rule{0.900pt}{0.116pt}}
\multiput(1378.00,455.17)(10.132,6.000){2}{\rule{0.450pt}{0.400pt}}
\multiput(1390.00,462.59)(1.033,0.482){9}{\rule{0.900pt}{0.116pt}}
\multiput(1390.00,461.17)(10.132,6.000){2}{\rule{0.450pt}{0.400pt}}
\multiput(1402.00,468.59)(1.123,0.482){9}{\rule{0.967pt}{0.116pt}}
\multiput(1402.00,467.17)(10.994,6.000){2}{\rule{0.483pt}{0.400pt}}
\multiput(1415.00,474.59)(1.033,0.482){9}{\rule{0.900pt}{0.116pt}}
\multiput(1415.00,473.17)(10.132,6.000){2}{\rule{0.450pt}{0.400pt}}
\multiput(1427.00,480.59)(1.033,0.482){9}{\rule{0.900pt}{0.116pt}}
\multiput(1427.00,479.17)(10.132,6.000){2}{\rule{0.450pt}{0.400pt}}
\put(1279,654){\makebox(0,0)[r]{Algorytm 2}}
\multiput(1299,654)(20.756,0.000){5}{\usebox{\plotpoint}}
\put(1399,654){\usebox{\plotpoint}}
\put(231,515){\usebox{\plotpoint}}
\put(231.00,515.00){\usebox{\plotpoint}}
\put(251.33,519.08){\usebox{\plotpoint}}
\put(271.71,522.93){\usebox{\plotpoint}}
\put(292.04,527.01){\usebox{\plotpoint}}
\put(312.38,531.09){\usebox{\plotpoint}}
\put(332.75,534.94){\usebox{\plotpoint}}
\put(353.09,539.01){\usebox{\plotpoint}}
\put(373.42,543.10){\usebox{\plotpoint}}
\put(393.86,546.64){\usebox{\plotpoint}}
\put(414.13,551.02){\usebox{\plotpoint}}
\put(434.46,555.12){\usebox{\plotpoint}}
\put(454.90,558.65){\usebox{\plotpoint}}
\put(475.17,563.03){\usebox{\plotpoint}}
\put(495.65,566.44){\usebox{\plotpoint}}
\put(515.96,570.66){\usebox{\plotpoint}}
\put(536.43,574.11){\usebox{\plotpoint}}
\put(556.70,578.45){\usebox{\plotpoint}}
\put(577.13,582.03){\usebox{\plotpoint}}
\put(597.48,586.08){\usebox{\plotpoint}}
\put(617.80,590.20){\usebox{\plotpoint}}
\put(638.25,593.71){\usebox{\plotpoint}}
\put(658.71,597.18){\usebox{\plotpoint}}
\put(678.99,601.50){\usebox{\plotpoint}}
\put(699.49,604.75){\usebox{\plotpoint}}
\put(719.76,609.13){\usebox{\plotpoint}}
\put(740.24,612.54){\usebox{\plotpoint}}
\put(760.66,616.16){\usebox{\plotpoint}}
\put(781.01,620.17){\usebox{\plotpoint}}
\put(801.48,623.58){\usebox{\plotpoint}}
\put(821.98,626.83){\usebox{\plotpoint}}
\put(842.43,630.36){\usebox{\plotpoint}}
\put(862.74,634.50){\usebox{\plotpoint}}
\put(883.22,637.87){\usebox{\plotpoint}}
\put(903.70,641.28){\usebox{\plotpoint}}
\put(924.19,644.57){\usebox{\plotpoint}}
\put(944.67,647.94){\usebox{\plotpoint}}
\put(965.14,651.36){\usebox{\plotpoint}}
\put(985.64,654.64){\usebox{\plotpoint}}
\put(1006.11,658.02){\usebox{\plotpoint}}
\put(1026.59,661.43){\usebox{\plotpoint}}
\put(1047.08,664.71){\usebox{\plotpoint}}
\put(1067.56,668.09){\usebox{\plotpoint}}
\put(1088.03,671.51){\usebox{\plotpoint}}
\put(1108.53,674.77){\usebox{\plotpoint}}
\put(1129.01,678.17){\usebox{\plotpoint}}
\put(1149.48,681.58){\usebox{\plotpoint}}
\put(1170.08,683.93){\usebox{\plotpoint}}
\put(1190.56,687.26){\usebox{\plotpoint}}
\put(1211.04,690.67){\usebox{\plotpoint}}
\put(1231.53,693.93){\usebox{\plotpoint}}
\put(1252.09,696.67){\usebox{\plotpoint}}
\put(1272.60,699.77){\usebox{\plotpoint}}
\put(1293.10,703.01){\usebox{\plotpoint}}
\put(1313.70,705.45){\usebox{\plotpoint}}
\put(1334.17,708.86){\usebox{\plotpoint}}
\put(1354.78,711.13){\usebox{\plotpoint}}
\put(1375.26,714.54){\usebox{\plotpoint}}
\put(1395.79,717.48){\usebox{\plotpoint}}
\put(1416.35,720.23){\usebox{\plotpoint}}
\put(1436.93,722.83){\usebox{\plotpoint}}
\put(1439,723){\usebox{\plotpoint}}
\put(231.0,131.0){\rule[-0.200pt]{0.400pt}{155.380pt}}
\put(231.0,131.0){\rule[-0.200pt]{291.007pt}{0.400pt}}
\put(1439.0,131.0){\rule[-0.200pt]{0.400pt}{155.380pt}}
\put(231.0,776.0){\rule[-0.200pt]{291.007pt}{0.400pt}}
\end{picture}

\caption{Wykres wydajnosci (zblizenie)}
\end{figure}

\section{Wnioski}

Wnioski z przeprowadzonej analizy są bardzo proste - algorytm podwajający wielkość
tablicy przy jej przepełnieniu i zwalniający gdy jedna-czwarta jest zajęta
jest o wiele bardziej wydajniejszym i efektywniejszym stosem. Trudność zauważenia
wykresu tego algorytmu z pewnością za tym przemawia. Drugi algorytm również
spełnia swoje zadanie, jednak prędko pnie się ku górze, co przemawia na jego 
nie korzyść. \\
Próbując odpowiedzieć na pytanie: "Dlaczego akurat ten algorytm jest lepszy?"
myślę, że zajmowanie nowego kawałka pamięci dla każdego nowego elemntu i
zwalnianie jej musi być bardzo obliczenio-chłonne, a przez to o wiele wolniejsze.
Za to rzadkie, acz konkretne zwiększanie pamięci i zwalnianie jej jest wyjściem
optymalnym, pozwalającym na ulepszenie działania mojego stosu.\\\\
Jednakże otrzymane wyniki budzą moje obawy, iż stworzony przeze mnie program
benchmarkujący nie działa odpowiednio. Wydaje się dla mnie dziwna aż taka
rozbieżność danych, a tym samym różnica w efektywności działania algorytmów.

\end{document}