\documentclass[a4paper,12pt]{article}

\usepackage[utf8]{inputenc}
\usepackage[T1]{fontenc}
\usepackage[MeX]{polski}
\usepackage[polish]{babel}
\usepackage[latin2]{inputenc}
\usepackage[T1]{fontenc}
\usepackage{graphicx}
\usepackage{subfig}

\begin{document}
{\raggedleft{}Krzysztof Kucharczyk}\\200401\\Wydział Elektroniki\\Kierunek AiR
\\PAMSI lab. czw. 10:00-13:15\\\\\\
\begin{center} 
	\textbf{Sprawozdanie z laboratorium nr 5\\(Ulepszenie algorytmu QuickSort)}
\end{center}

\section{Opis zadania}

Zadanie polegało na ulepszeniu algorytmu sortowania szybkiego w ten sposób, aby szansa na 
wystąpienie przypadku najgorszego była możliwie jak najniższa. 
\section{Realizacja}
W celu zrealizowania postawionego problemu dokonałem zmiany w kodzie programu, dzięki której piwot zostaje wylosowany
spośród zakresu możliwych danych. Dzięki takiemu zabiegowi szansa na otrzymanie kwadratowej złożoności obliczeniowej 
jest niezwykle niska, wręcz zaniedbywalna. W ten prosty sposób algorytm sortowania szybkiego staje się wydajniejszy.

\end{document}
